\documentclass[aspectratio=169]{beamer}
\usepackage{amsmath, amssymb, amsfonts, mathtools}
\usepackage{booktabs, graphicx}
\usepackage{hyperref}
\usepackage{fontspec}
\usepackage{siunitx}
\usepackage{physics}
\usepackage{caption}
\usepackage{listings}
\usepackage{tikz}
\usepackage{tabularx}
% \usepackage{ulem}
\usetikzlibrary{arrows.meta,positioning,shapes,fit,calc}
\usepackage{pgfplots}
\pgfplotsset{compat=1.18}
  \newcommand{\E}{\mathbb{E}}
\newcommand{\Var}{\mathrm{Var}}

%Для защит онлайн лучше использовать разрешение 16x9
%\documentclass[aspectratio=169]{beamer}

\input{preamble.tex}

% То, что в квадратных скобках, отображается внизу по центру каждого слайда.
\title[MyGym Finance]{Финансовый помощник для посетителей спортзалов}
\subtitle{Инвест-питч и бизнес-план}

% То, что в квадратных скобках, отображается в левом нижнем углу.
\institute[СПбГУ]{Экономико-правовые основы рынка}

% То, что в квадратных скобках, отображается в левом нижнем углу.
\author[Альшаеб Басель]{Альшаеб Басель, группа 24.М71-мм}

\begin{document}
{
\setbeamertemplate{footline}{}
% Лого университета или организации, отображается в шапке титульного листа
\begin{frame}
  \includegraphics[width=1.4cm]{pictures/SPbGU_Logo.png}
\vspace{-35pt}
\hspace{-10pt}
\begin{center}
   \begin{tabular}{c}
        \scriptsize{Санкт-Петербургский государственный университет} \\
        \scriptsize{Кафедра системного программирования}
    \end{tabular}
\titlepage
\end{center}

\btVFill

{\scriptsize
  % У научного руководителя должна быть указана научная степень
   \textbf{Преподаватель:} Старший преподаватель М.Х. Немешев \\
 }
\begin{center}
  \vspace{5pt}
  \scriptsize{Санкт-Петербург\\
                 2025}
  \end{center}

\end{frame}
}

% ---------- Проблема ----------
\begin{frame}{Проблема}
  \begin{itemize}
    \item Пользователи \textbf{переплачивают} за абонементы, добавки, персональные тренировки без прозрачного планирования.
    \item Нет \textbf{единого инструмента}, который связывает тренировки, питание и финансы.
    \item Фитнес-клубам сложно \textbf{удерживать клиентов} и делать релевантные апселлы.
  \end{itemize}
  \vspace{0.5em}
  \begin{block}{Воздействуем на когнитивный и эмоциональный уровни}
    Краткий путь по модели Левиджа–Штайнера: \textit{Осведомленность} $\to$ \textit{Знания} $\to$ \textit{Симпатия} $\to$ \textit{Предпочтение} $\to$ \textit{Покупка}.
  \end{block}
\end{frame}

% % ---------- Решение ----------
\begin{frame}{Решение: MyGym Finance}
  \begin{columns}[T]
    \begin{column}{0.56\textwidth}
      \begin{itemize}
        \item Мобильное приложение: бюджет + тренировки + аналитика ценности.
        \item ИИ-план на месяц: как сэкономить 15–25\% без потери качества тренировок.
        \item Интеграции: абонемент, CRM клуба, кэшбэк-партнёры.
        \item Напоминания: «Смените на квартальный — экономия 20\%».
        \item Геймификация: очки за дисциплину, челленджи.
      \end{itemize}
    \end{column}
    \begin{column}{0.42\textwidth}
      % \begin{center}
      %   \includegraphics[width=\linewidth]{app-mockup-placeholder.pdf}\\[0.5em]
      %   \small \emph{Мокап экрана: дешборд экономии и прогресса.}
      % \end{center}
    \end{column}
  \end{columns}
\end{frame}



% % ---------- Рынок и аудитория ----------
\begin{frame}{Рынок и аудитория}
  \begin{itemize}
    \item ЦА: 18–40 лет, активные посетители залов, студенты, молодые специалисты.
    \item Тренд: Digital Wellness и финтех растут двузначными темпами.
    \item TAM $\sim$ рынок фитнеса ($\approx\$80$ млрд), SAM — цифровой велнесс ($\approx\$10$ млрд), SOM — локальные рынки СНГ/Европы.
  \end{itemize}
  \vspace{0.6em}
  \begin{tikzpicture}[every node/.style={font=\small}]
    \node[draw,rounded corners,inner sep=6pt,fill=blue!7] (tam) {TAM: Фитнес $ \approx \$80$ млрд};
    \node[draw,rounded corners,inner sep=6pt,fill=white,below=4mm of tam] (sam) {SAM: Digital Wellness $ \approx \$10$ млрд};
    \node[draw,rounded corners,inner sep=6pt,fill=white,below=4mm of sam] (som) {SOM: Стартовые регионы (СНГ/EU)};
    \draw[-{Latex[length=3mm]}] (tam) -- (sam);
    \draw[-{Latex[length=3mm]}] (sam) -- (som);
  \end{tikzpicture}
\end{frame}


% ---------- Slide 4: Конкуренты ----------
\begin{frame}{Конкуренты и наше преимущество}
  \small
  \begin{tabularx}{\linewidth}{>{\bfseries}l l l X}
    \toprule
    Конкурент & Фокус & Слабость & Наше преимущество \\
    \midrule
    Zenmoney & Финансы & Нет связки с залом & Единая аналитика фитнес+бюджет \\
    MyFitnessPal & Фитнес/калории & Нет фин.аналитики & Финансовая ценность тренировок \\
    Клубные приложения & Лояльность & Нет ИИ-экономии & Персональные советы и кэшбэк \\
    \bottomrule
  \end{tabularx}
  \vspace{0.5em}

  \begin{block}{Moat (барьеры копирования)}
    Интеграции с локальными клубами, партнёрская сеть кэшбэка, анонимная агрегированная аналитика для B2B.
  \end{block}
\end{frame}



% % ---------- Бизнес-модель и цены ----------
\begin{frame}{Бизнес-модель и ценообразование}
  \textbf{Freemium}: базовые функции бесплатно, \textbf{Premium} — 299 ₽/мес. \par
  B2B для клубов: 1500 ₽/локацию/мес.

  \medskip
  \textbf{Принципы из курса «Ценообразование»}:
  \begin{itemize}
    \item Ориентация на спрос + конкуренцию; «справедливая цена» и ценность.
    \item Дифференциация: студенческие/семейные планы, квартальная экономия.
    \item \textit{Value-based}: считаем экономию клиента и соотносим с ценой Premium.
  \end{itemize}

  \medskip
  \begin{block}{Пример value-case}
    Средний пользователь экономит 600–1200 ₽/мес → Premium окупается 2–4x.
  \end{block}
\end{frame}




% % ---------- Продвижение ----------
\begin{frame}{Продвижение ПО (инвест-фокус)}
  \textbf{Воронка (Роджерс):} Осведомлённость → Интерес → Оценка → Испытание → Принятие.
  \begin{itemize}
    \item Awareness: Reels/Shorts с тренерами, UGC-челленджи.
    \item Trial: 30 дней Premium, геймифицированные недели дисциплины.
    \item PR: локальные турнирчики, университетские залы, коллаборации с нутрициологами.
    \item CRM: сегментация пушей, «усовершенствованные стимулы» (персональные скидки).
  \end{itemize}
\end{frame}


% % ---------- Финплан и инвестиции ----------
\begin{frame}{Финплан и запрос инвестиций}
  \begin{columns}[T]
    \begin{column}{0.54\textwidth}
      \textbf{MVP (3 мес)}: ~600 тыс. ₽ \newline
      Разработка, дизайн, бэк+мобайл, интеграции, тесты.

      \medskip
      \textbf{12 мес прогноз (пример):}
      \begin{itemize}
        \item 10k пользователей, конверсия Premium 10\%.
        \item Выручка: ~3.0 млн ₽. B2B: ~0.9 млн ₽.
        \item Итого: ~3.9 млн ₽ (без учёта комиссий).
      \end{itemize}
    \end{column}
    \begin{column}{0.44\textwidth}
      \textbf{Запрос: 0.8–1.2 млн ₽} \newline
      на маркетинг и масштабирование.
      \medskip

      \begin{block}{Окупаемость}
        Цель: ROI 150\% за 18 мес (ориентир).
      \end{block}
    \end{column}
   \end{columns}
   \tiny \emph{Числа примерные для учебного питча; уточняются после маркет.исследований.}
 \end{frame}



 % ---------- Roadmap ----------
\begin{frame}{Roadmap}
  \begin{tabularx}{\linewidth}{>{\bfseries}l X}
    \toprule
    Q1 & MVP: бюджет+тренировки, базовая аналитика, 2 пилотных клуба \\
    Q2 & Запуск, лид-магниты, партнёрства, кэшбэк v1 \\
    Q3 & Premium AI, B2B-дашборды, региональное расширение \\
    Q4 & Масштабирование, новые рынки, локализация EU \\
    \bottomrule
  \end{tabularx}
\end{frame}


% ---------- Технологии ----------
\begin{frame}{Технологии и интеграции}
  \begin{itemize}
    \item Мобильные клиенты (iOS/Android), бэкэнд (REST/GraphQL), интеграции с CRM фитнес-клубов.
    \item МЛ-модуль: персональные советы экономии, прогнозирование затрат/посещаемости.
    \item Безопасность: OAuth2/Sanctum, шифрование платежных токенов (PCI DSS партнёра).
  \end{itemize}
\end{frame}




% ---------- Команда ----------
\begin{frame}{Команда}
  \begin{columns}[T]
    \begin{column}{0.48\textwidth}
      \begin{itemize}
        \item \textbf{CEO/Founder}: видение продукта, партнёрства.
        \item \textbf{CTO}: архитектура, масштабирование, качество.
        \item \textbf{Marketing Lead}: рост, PR, инфлюенсеры.
      \end{itemize}
    \end{column}
    \begin{column}{0.48\textwidth}
      \begin{itemize}
        \item \textbf{ML Engineer}: рекомендации, аналитика.
        \item \textbf{Mobile Lead}: UX, геймификация, ретеншн.
      \end{itemize}
    \end{column}
  \end{columns}
\end{frame}



% ---------- Риски и меры ----------
\begin{frame}{Ключевые риски и меры}
  \begin{itemize}
    \item \textbf{Низкая конверсия Premium} → A/B ценообразования, бандлы, квартальные планы.
    \item \textbf{Интеграции с клубами} → пилоты, стандартный SDK, SLA.
    \item \textbf{Конкуренция} → фокус на value-based советах и локальные партнёрства.
  \end{itemize}
\end{frame}


% ---------- Финал ----------
\begin{frame}{Почему инвестировать в MyGym Finance}
  \begin{itemize}
      \item \textbf{Новый сегмент}: стык финтеха и фитнеса с реальной экономией для пользователя.
      \item \textbf{Модель} с мульти-доходностью: B2C Premium + B2B/партнёрки.
      \item \textbf{Команда} с опытом бэкенда, мобайла и интеграций.
  \end{itemize}
  \vspace{0.5em}
  \begin{center}
    \Large Тренируйся умно — трать ещё умнее.
  \end{center}
\end{frame}

% ---------- Q&A ----------
\begin{frame}
  Вопросы?
\end{frame}


\end{document}
