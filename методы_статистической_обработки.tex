\documentclass{beamer}
\usepackage{amsmath, amssymb, amsfonts, mathtools}
\usepackage{booktabs, graphicx}
\usepackage{hyperref}
\usepackage{siunitx}
\usepackage{physics}
\usepackage{caption}
\usepackage{listings}
\newcommand{\E}{\mathbb{E}}
\newcommand{\Var}{\mathrm{Var}}

%Для защит онлайн лучше использовать разрешение 16x9
%\documentclass[aspectratio=169]{beamer}

\input{preamble.tex}

% То, что в квадратных скобках, отображается внизу по центру каждого слайда.
\title[Регрессия и корреляция в Python]{Регрессионный анализ и корреляция\\(живая демонстрация в Python)}

% То, что в квадратных скобках, отображается в левом нижнем углу.
\institute[СПбГУ]{\\ Курс «Методы статистической обработки информации»}

% То, что в квадратных скобках, отображается в левом нижнем углу.
\author[Альшаеб Басель]{Альшаеб Басель, группа 24.М71-мм}

\begin{document}
{
\setbeamertemplate{footline}{}
% Лого университета или организации, отображается в шапке титульного листа
\begin{frame}
  \includegraphics[width=1.4cm]{pictures/SPbGU_Logo.png}
\vspace{-35pt}
\hspace{-10pt}
\begin{center}
   \begin{tabular}{c}
        \scriptsize{Санкт-Петербургский государственный университет} \\
        \scriptsize{Кафедра системного программирования}
    \end{tabular}
\titlepage
\end{center}

\btVFill

{\scriptsize
  % У научного руководителя должна быть указана научная степень
   \textbf{Преподаватель:} д.ф.-м.н. профессор Н.К. Кривулин \\
 }
\begin{center}
  \vspace{5pt}
  \scriptsize{Санкт-Петербург\\
                 2025}
  \end{center}

\end{frame}
}


% ====== СЛАЙД: ЦЕЛИ И ФОРМАТ ======
\begin{frame}{Формат экзаменационной презентации}
\textbf{Цель:} продемонстрировать решение задач статистического анализа на выбранном инструменте (Python: pandas, numpy, matplotlib, statsmodels).\\[4pt]
\textbf{Требование к структуре:}
\begin{itemize}
  \item краткое введение в инструмент;
  \item демонстрация решения задачи на примере реальных/синтетических данных;
  \item интерпретация результатов и выводы.
\end{itemize}
\textit{Соответствует общему плану презентации из методических указаний.}
\end{frame}



% ====== СЛАЙД: ПЛАН ======
\begin{frame}{План выступления}
\begin{enumerate}
  \item Постановка задачи и данные.
  \item Краткое введение в Python-инструментарий.
  \item Теория: корреляция и линейная регрессия.
  \item Демонстрация:
  \begin{itemize}
    \item описательная статистика и корреляции;
    \item модель OLS, сводка, проверка гипотез;
    \item диагностика остатков, доверительные и предсказательные интервалы;
    \item краткая связь с ANOVA (F-тест).
  \end{itemize}
  \item Выводы и ограничения.
\end{enumerate}
\end{frame}


% ====== СЛАЙД: ПОСТАНОВКА ЗАДАЧИ ======
\begin{frame}{Постановка задачи}
\textbf{Задача:} изучить линейную связь между признаками и целевой переменной и построить модель для предсказания.\\[6pt]
\textbf{Пример:} прогноз цены по площади и числу комнат (синтетический набор данных, размер $n=200$).\\[6pt]
\textbf{Почему линейная регрессия и корреляция?}
\begin{itemize}
  \item простые модели;
  \item тесная связь с проверкой гипотез и доверительными интервалами;
  \item соответствуют темам курса (оценивание, гипотезы, корреляция, регрессия, ANOVA).
\end{itemize}
\end{frame}

% ====== СЛАЙД: ИНСТРУМЕНТ ======
\begin{frame}{Инструментарий Python для демонстрации}
\begin{itemize}
  \item \texttt{pandas}: загрузка/предобработка данных, таблицы;
  \item \texttt{numpy}: работа с массивами, генерация данных;
  \item \texttt{matplotlib}: базовые графики;
  \item \texttt{statsmodels}: регрессия OLS, сводка, интервал оценки и предсказания.
\end{itemize}
\textbf{Запуск:} Jupyter/Colab/VS~Code.\\
\textbf{Установка:} \texttt{pip install pandas numpy matplotlib statsmodels}
\end{frame}

% ====== СЛАЙД: ТЕОРИЯ КОРРЕЛЯЦИИ ======
\begin{frame}{Коэффициент корреляции Пирсона}
Для двух выборок $(x_i)$ и $(y_i)$, $i=1,\dots,n$, выборочная ковариация:
\[
s_{XY}=\frac{1}{n}\sum_{i=1}^n (x_i-\bar x)(y_i-\bar y),
\]
выборочные дисперсии $s_X^2, s_Y^2$, коэффициент корреляции:
\[
r_{XY}=\frac{\sum_{i=1}^n (x_i-\bar x)(y_i-\bar y)}
{\sqrt{\sum_{i=1}^n (x_i-\bar x)^2}\sqrt{\sum_{i=1}^n (y_i-\bar y)^2}}\in[-1,1].
\]
Интерпретация: сила \emph{линейной} связи (не причинность). При $|r|\approx 1$ зависимость близка к линейной.
\end{frame}


% ====== СЛАЙД: ТЕОРИЯ РЕГРЕССИИ ======
\begin{frame}{Линейная регрессия (многомерная)}
Модель:
\[
y=\beta_0+\beta_1x_1+\dots+\beta_px_p+\varepsilon,\quad \E[\varepsilon]=0,\ \Var(\varepsilon)=\sigma^2.
\]
Оценивание $\beta$ методом наименьших квадратов (МНК):
\[
\widehat\beta=\arg\min_\beta \sum_{i=1}^n (y_i-\beta_0-\sum_{j=1}^p\beta_j x_{ij})^2.
\]
Качество: $R^2$, скорректированный $R^2$. Проверка гипотез: $H_0:\beta_j=0$. Доверительные интервалы для коэффициентов и предсказаний.
\end{frame}

% ====== СЛАЙД: СВЯЗЬ С ГИПОТЕЗАМИ И ИНТЕРВАЛАМИ ======
\begin{frame}{Гипотезы и интервальные оценки в регрессии}
\begin{itemize}
  \item \textbf{Проверка гипотез} о параметрах: $H_0:\beta_j=0$ (t-стат., p-value).
  \item \textbf{F-тест} значимости всей модели (связь с ANOVA).
  \item \textbf{Доверительные интервалы} для параметров (оценки и их точность).
  \item \textbf{Предсказательные интервалы} для новых наблюдений (учитывают шум).
\end{itemize}
\end{frame}


% ====== СЛАЙД: ДЕМО — ДАННЫЕ ======
\begin{frame}[fragile]{Демонстрация: генерация данных}
\begin{lstlisting}[language=Python]
import numpy as np, pandas as pd, matplotlib.pyplot as plt
import statsmodels.api as sm; np.random.seed(42)

n = 200
area  = np.random.uniform(30, 150, n)   # площадь
rooms = np.random.uniform(1, 5, n)      # комнаты
eps   = np.random.normal(0, 15, n)
price = 20 + 0.8*area + 10*rooms + eps  # цена

df = pd.DataFrame({"area": area, "rooms": rooms, "price": price})
df.head()
\end{lstlisting}
\end{frame}


% ====== СЛАЙД: ДЕМО — ОПИСАТЕЛЬНАЯ И КОРРЕЛЯЦИЯ ======
\begin{frame}[fragile]{Демонстрация: описательная статистика и корреляция}
\begin{lstlisting}[language=Python]
print(df.describe(numeric_only=True))
print("\nКорреляции:\n", df.corr(numeric_only=True))

plt.figure()
plt.scatter(df["area"], df["price"])
plt.xlabel("area"); plt.ylabel("price")
plt.title("Связь area–price"); plt.show()
\end{lstlisting}
\textit{Интерпретация:} знак и величина $r$ показывают направление и силу линейной связи.
\end{frame}

% ====== СЛАЙД: ДЕМО — МОДЕЛЬ OLS ======
\begin{frame}[fragile]{Демонстрация: модель OLS и сводка}
\begin{lstlisting}[language=Python]
X = sm.add_constant(df[["area","rooms"]])
y = df["price"]
model = sm.OLS(y, X).fit()
print(model.summary())
\end{lstlisting}
\textbf{Смотрим:} коэффициенты $\hat\beta$, их стандартные ошибки, t-статистики, p-values, $R^2$, F-статистику.\\
\textit{Вывод:} если p-value для признака $<0.05$, вклад статистически значим.
\end{frame}

% ====== СЛАЙД: ДЕМО — ДИАГНОСТИКА ОСТАТКОВ ======
\begin{frame}[fragile]{Демонстрация: диагностика остатков}
\begin{lstlisting}[language=Python]
resid, fitted = model.resid, model.fittedvalues

plt.figure()
plt.scatter(fitted, resid); plt.axhline(0)
plt.xlabel("Предсказания"); plt.ylabel("Остатки")
plt.title("Остатки vs предсказания"); plt.show()

sm.qqplot(resid, line="45"); plt.title("QQ-plot остатков"); plt.show()
\end{lstlisting}
\textit{Интерпретация:} отсутствие «веера» (гомоскедастичность), QQ-plot близок к прямой (нормальность остатков).
\end{frame}

% ====== СЛАЙД: ДЕМО — ИНТЕРВАЛЫ ======
\begin{frame}[fragile]{Демонстрация: доверительные и предсказательные интервалы}
\begin{lstlisting}[language=Python]
print("ДИ для коэффициентов:\n", model.conf_int())

new_obs = pd.DataFrame({"area":[80,120], "rooms":[2,3]})
new_X   = sm.add_constant(new_obs)
pred    = model.get_prediction(new_X)
print(pred.summary_frame(alpha=0.05))  # mean_ci_lower/upper, obs_ci_lower/upper
\end{lstlisting}
\textbf{Важно:} CI для среднего предсказания уже, чем PI для индивидуального наблюдения.
\end{frame}


% ====== СЛАЙД: СВЯЗЬ С ANOVA ======
\begin{frame}{Связь с дисперсионным анализом (ANOVA)}
F-тест из сводки \texttt{summary()} проверяет $H_0$: «все регрессионные коэффициенты (кроме константы) равны нулю».\\[6pt]
\textit{Если p-value(F) $<0.05$}, модель в целом значима. Это соответствует идее сравнения объяснённой и остаточной дисперсий в ANOVA.
\end{frame}

% ====== СЛАЙД: ОГРАНИЧЕНИЯ И УЛУЧШЕНИЯ ======
\begin{frame}{Ограничения и улучшения}
\begin{itemize}
  \item Линейность связи: при нелинейности — полиномиальные/лог-преобразования.
  \item Гомоскедастичность: при «веере» — робастные ошибки (Newey–West), взвешенный МНК.
  \item Мультиколлинеарность: регуляризация (Ridge/Lasso).
  \item Валидация: train/test, кросс-валидация, устойчивость результатов.
\end{itemize}
\end{frame}

% ====== СЛАЙД: ВЫВОДЫ ======
\begin{frame}{Выводы}
\begin{itemize}
  \item Корреляция и регрессия дают наглядные и интерпретируемые результаты.
  \item Python + statsmodels позволяют быстро пройти путь: данные $\to$ модель $\to$ гипотезы $\to$ интервалы.
  \item Диагностика остатков — обязательна для корректной интерпретации.
\end{itemize}
Код демонстрации можно поместить в Jupyter/Colab и приложить к сдаче.
\end{frame}

% ====== СЛАЙД: ИСТОЧНИКИ ======
\begin{frame}{Источники (курс)}
\begin{itemize}
  \item Конспект курса «Методы статистической обработки информации» (Кривулин Н.К.): разделы про оценивание, корреляцию, регрессию, ANOVA, интервальные оценки.
  \item Методические рекомендации по формату презентации и темам.
\end{itemize}
\end{frame}

\end{document}
