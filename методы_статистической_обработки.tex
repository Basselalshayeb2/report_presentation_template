\documentclass[aspectratio=169]{beamer}
\usepackage{amsmath, amssymb, amsfonts, mathtools}
\usepackage{booktabs, graphicx}
\usepackage{hyperref}
\usepackage{siunitx}
\usepackage{physics}
\usepackage{caption}
\usepackage{listings}
\usepackage{minted}
\newcommand{\E}{\mathbb{E}}
\newcommand{\Var}{\mathrm{Var}}

%Для защит онлайн лучше использовать разрешение 16x9
% \documentclass[aspectratio=169]{beamer}

\input{preamble.tex}

% То, что в квадратных скобках, отображается внизу по центру каждого слайда.
\title[Корреляционный анализ в Python]{Корреляционный анализ в Python\\(Pearson, Spearman, p-value, доверительные интервалы)}

% То, что в квадратных скобках, отображается в левом нижнем углу.
\institute[СПбГУ]{\\ Курс «Методы статистической обработки информации»}

% То, что в квадратных скобках, отображается в левом нижнем углу.
\author[Альшаеб Басель]{Альшаеб Басель, группа 24.М71-мм}

\begin{document}
{
\setbeamertemplate{footline}{}
% Лого университета или организации, отображается в шапке титульного листа
\begin{frame}
  \includegraphics[width=1.4cm]{pictures/SPbGU_Logo.png}
\vspace{-35pt}
\hspace{-10pt}
\begin{center}
   \begin{tabular}{c}
        \scriptsize{Санкт-Петербургский государственный университет} \\
        \scriptsize{Кафедра системного программирования}
    \end{tabular}
\titlepage
\end{center}

\btVFill

{\scriptsize
  % У научного руководителя должна быть указана научная степень
   \textbf{Преподаватель:} д.ф.-м.н. профессор Н.К. Кривулин \\
 }
\begin{center}
  \vspace{5pt}
  \scriptsize{Санкт-Петербург\\
                 2025}
  \end{center}

\end{frame}
}

%
\begin{frame}{Введение}
  \begin{itemize}
    \item \textbf{Цель:} показать, как выполнить корреляционный анализ в Python и корректно интерпретировать результаты для принятия решений на основе данных
    \item \textbf{Показатели:}
    \begin{itemize}
      \item \textbf{коэффициент Пирсона}: для линейной связи
      \item \textbf{коэффициент Спирмена}: для монотонной связи и в ситуациях с выбросами
    \end{itemize}
  \end{itemize}
  \end{frame}

% ====== СЛАЙД: ПЛАН ======
% \begin{frame}{План выступления}
% \begin{enumerate}
%   \item Постановка задачи и данные.
%   \item Краткое введение в инструмент (Python, экосистема, интерфейсы).
%   \item Демонстрация на примерах корреляционного анализа (Pearson/Spearman, p-value, CI).
%   \item Выводы и ограничения.
% \end{enumerate}
% \end{frame}

% ================================ СЛАЙД: Постановка задачи и данные ================================
\begin{frame}[fragile]{Постановка задачи и данные}
  \begin{itemize}
    \item \textbf{Задача:}
    \begin{itemize}
      \item посчитать значения корреляций
      \item проверить их статистическую значимость
      \item построить 95\% доверительные интервалы
      \item визуализировать результаты матрицей и scatter-графиками
      \item выделить признаки с наиболее устойчивой связью
    \end{itemize}
    % \item Постановка задачи: изучение зависимости между признаками и целевой переменной в медицине.
    \item \textbf{Данные:} датасет из 442 наблюдений и 10 признаков из
      \begin{minted}{python}
        from sklearn.datasets import load_diabetes
        data = load_diabetes()
        X, y = data.data, data.target
      \end{minted}
    \end{itemize}
\end{frame}

% ================================ СЛАЙД: Python ================================
\begin{frame}{Python}
  \textbf{Язык:} Python 3.x:
  \begin{itemize}
    \item высокоуровневый интерпретируемый язык программирования, используемый в статистическом анализе из-за простоты синтаксиса и огромного количества библиотек для работы с данными
  \end{itemize}
  \textbf{Библиотеки:} \texttt{pandas}, \texttt{numpy}, \texttt{scipy}, \texttt{matplotlib}.\\[4pt]
  \textbf{Почему Python?}
  \begin{itemize}
    \item Широкая экосистема статистики.
    \item Быстрый путь: данные $\rightarrow$ анализ $\rightarrow$ графики
  \end{itemize}
  \textbf{Где запускать:} Jupyter Notebook/Colab/VS Code
\end{frame}

% ================================ СЛАЙД: Выбор среды разработки: почему Google Colab ================================
\begin{frame}{Выбор среды разработки: почему Google Colab}

  {\scriptsize
  \begin{table}[h!]
    \centering
    \begin{tabular}{|l|c|c|c|}
      \hline
      \textbf{Критерий} & \textbf{Colab} & \textbf{Jupyter Notebook} & \textbf{VS Code} \\ \hline
      Установка          & Не требуется (облако) & Требуется локально & Требуется локально \\ \hline
      Мощность ресурсов  & GPU/TPU (ограниченно) & Завист от ПК        & Завист от ПК       \\ \hline
      Портативность      & Максимальная          & Средняя             & Средняя            \\ \hline
      Совместимость      & Браузер               & Локальная конфигурация & Расширения/настройка \\ \hline
      Работа с данными   & Лёгкий импорт из Drive & Локальные файлы     & Проекты/репозитории \\ \hline
      Стоимость          & Бесплатно             & Бесплатно           & Бесплатно          \\ \hline
    \end{tabular}
  \end{table}
  }
  \vspace{0.4cm}

  \textbf{Почему выбрал Google Colab:}
  \begin{itemize}
    \item не требует установки — работает в браузере на любом устройстве
    \item бесплатных ресурсов достаточно для задач статистики
    \item легко подключать данные (Google Drive, GitHub)
    \item позволяет открывать блокнот везде — дома, в университете, на работе
  \end{itemize}
\end{frame}

% ================================ СЛАЙД: Структура демо ================================
\begin{frame}[fragile]{Структура демо}
  \begin{itemize}
    \item Импорт и загрузка данных
    \item Описательная статистика, матрица корреляций
    \item Pearson/Spearman: r, p-value, доверительные интервалы (Fisher z-transformation)
    \item Визуализация: scatter, тепловая карта
  \end{itemize}
\end{frame}


% ================================ СЛАЙД: Виды зависимости ================================
\begin{frame}{Корреляции: виды зависимости}
  \textbf{Корреляционный анализ}: изучает меры связи и позволяет проверять гипотезы о наличии/отсутствии связи \\
  \vspace{0.5cm}

  \begin{itemize}
  \item \textbf{Функциональная}: $Y=f(X)$
  \item \textbf{Стохастическая зависимость}: изменение $X$ влечёт изменение $Y$ по вероятностному закону
  \item \textbf{Независимость}: $P\{X\in A, Y\in B\}=P\{X\in A\}P\{Y\in B\}$
\end{itemize}
\end{frame}


% ================================ СЛАЙД: Ковариация и корреляция ================================
\begin{frame}{Коэффициент Пирсона}
  \textbf{Коэффициент Пирсона} - показывает, насколько две переменные изменяются вместе по прямой (линейно). % space before covariance

  \vspace{0.5cm}
  Ковариация: \[
  \mathrm{Cov}(X,Y) = E[XY] - E[X] \cdot E[Y]
  \]
  \vspace{0.5cm}
  Коэффициент Пирсона:  \[
  \rho = \dfrac{\mathrm{Cov}(X,Y)}{\sigma_X \sigma_Y} \in [-1,1]
  \]
  \textbf{Свойства:}
  \begin{itemize}
    \item не меняется, если данные смещать или умножать на константу
    \item $\rho = \pm 1 \Leftrightarrow Y = \beta_0 + \beta_1 X$
  \end{itemize}
  \textbf{Корреляция:} показывает связь, но не объясняет причину
\end{frame}


\begin{frame}{Коэффициент Спирмена}
  \textbf{Коэффициент Спирмена} измеряет монотонность связи двух переменных
  Основан на взаимном ранжировании, а не на абсолютных значениях

  \vspace{0.5cm}
  Формула (для данных без совпадающих рангов):
  \[
    \rho_s = 1 - \frac{6 \sum d_i^2}{n(n^2 - 1)}
  \]
  где $d_i$ — разность рангов.
\end{frame}


\begin{frame}{Pearson vs. Spearman}
\begin{tabular}{@{}p{0.32\linewidth}p{0.28\linewidth}p{0.28\linewidth}@{}}
\toprule
 & \textbf{Pearson $r$} & \textbf{Spearman $\rho_s$} \\
\midrule
Связь & Линейная & Монотонная (по рангам) \\
Выбросы & Чувствителен & Робастность выше \\
Данные & Интервальные/отношений & Порядковые/монотонные \\
\bottomrule
\end{tabular}
\end{frame}

\begin{frame}{Данные для демонстрации}
  \textbf{Набор данных:} \texttt{sklearn.datasets.load\_diabetes()} \\[0.2cm]

  Медицинский датасет из 442 наблюдений с непрерывными признаками:
  \begin{itemize}
    \item возраст,
    \item индекс массы тела (ИМТ),
    \item артериальное давление (АД),
    \item показатели крови и др.
  \end{itemize}

  Целевая переменная \texttt{target} отражает \textbf{прогресс заболевания}. \\[0.2cm]

  Набор удобен для:
  \begin{itemize}
    \item анализа корреляций,
    \item построения матрицы корреляций,
    \item парных scatter-графиков,
    \item проверки статистической значимости зависимостей.
  \end{itemize}
\end{frame}

\begin{frame}{Используемые библиотеки}
  \begin{itemize}
    \item \textbf{NumPy:} базовая библиотека для работы с массивами и матрицами, быстрые численные операции
    \item \textbf{Pandas:} удобная работа с табличными данными (датафреймы, очистка, агрегации, описательная статистика)
    \item \textbf{Matplotlib:} построение графиков (линий, точек, гистограмм, матриц корреляций)
    \item \textbf{SciPy:} функции статистики (p-value, критерии значимости, интервалы, распределения)
    \item \textbf{Scikit-learn:} загрузка датасетов, предобработка, метрики, простые модели машинного обучения
  \end{itemize}
\end{frame}


\begin{frame}[fragile]{Импорт и подготовка данных}

  \begin{minted}[fontsize=\footnotesize, linenos]{python}
  import numpy as np
  import pandas as pd
  import matplotlib.pyplot as plt
  from scipy import stats
  from sklearn.datasets import load_diabetes

  data = load_diabetes()

  X = pd.DataFrame(data.data, columns=data.feature_names)
  y = pd.Series(data.target, name="target")

  df = pd.concat([X, y], axis=1)

  df.head()
  df.describe()
  \end{minted}

\end{frame}


\begin{frame}[fragile]{Матрица корреляций (Pearson)}
\begin{minted}[fontsize=\footnotesize, linenos]{python}

corr_pearson = df.corr(numeric_only=True)
print(corr_pearson.round(3))

plt.figure()
plt.imshow(corr_pearson, vmin=-1, vmax=1)
plt.colorbar(); plt.title("Correlation matrix (Pearson)")
plt.xticks(range(len(corr_pearson.columns)), corr_pearson.columns, rotation=90)
plt.yticks(range(len(corr_pearson.index)), corr_pearson.index)
plt.tight_layout(); plt.show()
\end{minted}
Интерпретация: пары с $|r|\gtrsim 0.3$–$0.4$ — кандидаты на значимую линейную связь
\end{frame}


\begin{frame}[fragile]{Pearson: $r$ и p-value для пары признаков}
\begin{minted}[fontsize=\footnotesize, linenos]{python}
x = df["bmi"]        # индекс массы тела
y = df["target"]     # прогресс заболевания
r_xy, p_xy = stats.pearsonr(x, y)
print(f"Pearson r(bmi, target) = {r_xy:.3f}, p-value = {p_xy:.3g}")
\end{minted}
Гипотеза $H_0{:}\ \rho=0$. Если $p<0.05$, связь статистически значима. Знак $r$ укажет направление связи
\end{frame}

\begin{frame}[fragile]{Spearman (ранговая корреляция)}
\begin{minted}[fontsize=\footnotesize, linenos]{python}
corr_spearman = df.corr(numeric_only=True, method="spearman")
print(corr_spearman.round(3))

r_s, p_s = stats.spearmanr(df["bmi"], df["target"])
print(f"Spearman rho(bmi, target) = {r_s:.3f}, p-value = {p_s:.3g}")
\end{minted}
\end{frame}

\begin{frame}[fragile]{Доверительный интервал для $\rho$ (Fisher z-transformation)}
\begin{minted}[fontsize=\footnotesize, linenos]{python}
def fisher_ci(r, n, alpha=0.05):
    z = 0.5*np.log((1+r)/(1-r))
    z_se = 1/np.sqrt(n-3)
    z_crit = stats.norm.ppf(1-alpha/2)
    z_lo, z_hi = z - z_crit*z_se, z + z_crit*z_se
    to_r = lambda z: (np.exp(2*z)-1)/(np.exp(2*z)+1)
    return to_r(z_lo), to_r(z_hi)

n = len(df)
lo, hi = fisher_ci(r_xy, n)
print(f"95% CI for rho (bmi,target): [{lo:.3f}, {hi:.3f}]")
\end{minted}
Интерпретация: интервал не пересекает 0 $\Rightarrow$ линейная связь статистически значима
\end{frame}

\begin{frame}[fragile]{Визуализация: scatter-график}
\begin{minted}[fontsize=\footnotesize, linenos]{python}
plt.figure()
plt.scatter(df["bmi"], df["target"])
plt.xlabel("bmi"); plt.ylabel("target")
plt.title("bmi vs target (scatter)")
plt.show()
\end{minted}
Цель: определить вид связи (линейная/нелинейная), наличие выбросов
\end{frame}

% ====== APPLICATIONS / LIMITS ======
% \begin{frame}{Применение и ограничения}
% \textbf{Где полезно:}
% \begin{itemize}
%   \item Быстрый отбор признаков (feature screening) перед моделированием
%   \item Диагностика мультиколлинеарности (высокие межпризнаковые корреляции)
% \end{itemize}
% \textbf{Предосторожности:}
% \begin{itemize}
%   \item Pearson отражает \textbf{линейную} связь; Spearman — \textbf{монотонную}
%   \item Выбросы и неоднородная дисперсия искажают $r$ (используйте визуализацию)
%   \item Корреляция не объясняет причину
%   \item Множественные проверки $\Rightarrow$ корректировки (Bonferroni/FDR)
% \end{itemize}
% \end{frame}

\begin{frame}{Итоги}
\begin{itemize}
  \item Корреляция — простая и наглядная мера связи
  \item Python (\texttt{pandas}, \texttt{scipy}) даёт мгновенный расчёт $r$, p-value и CI
  \item Визуализация обязательна для корректной интерпретации
\end{itemize}
\end{frame}

\end{document}
