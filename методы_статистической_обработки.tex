\documentclass{beamer}
\usepackage{amsmath, amssymb, amsfonts, mathtools}
\usepackage{booktabs, graphicx}
\usepackage{hyperref}
\usepackage{siunitx}
\usepackage{physics}
\usepackage{caption}
\usepackage{listings}
\newcommand{\E}{\mathbb{E}}
\newcommand{\Var}{\mathrm{Var}}

%Для защит онлайн лучше использовать разрешение 16x9
%\documentclass[aspectratio=169]{beamer}

\input{preamble.tex}

% То, что в квадратных скобках, отображается внизу по центру каждого слайда.
\title[Регрессия и корреляция в Python]{Регрессионный анализ и корреляция\\(живая демонстрация в Python)}

% То, что в квадратных скобках, отображается в левом нижнем углу.
\institute[СПбГУ]{\\ Курс «Методы статистической обработки информации»}

% То, что в квадратных скобках, отображается в левом нижнем углу.
\author[Альшаеб Басель]{Альшаеб Басель, группа 24.М71-мм}

\begin{document}
{
\setbeamertemplate{footline}{}
% Лого университета или организации, отображается в шапке титульного листа
\begin{frame}
  \includegraphics[width=1.4cm]{pictures/SPbGU_Logo.png}
\vspace{-35pt}
\hspace{-10pt}
\begin{center}
   \begin{tabular}{c}
        \scriptsize{Санкт-Петербургский государственный университет} \\
        \scriptsize{Кафедра системного программирования}
    \end{tabular}
\titlepage
\end{center}

\btVFill

{\scriptsize
  % У научного руководителя должна быть указана научная степень
   \textbf{Преподаватель:} д.ф.-м.н. профессор Н.К. Кривулин \\
 }
\begin{center}
  \vspace{5pt}
  \scriptsize{Санкт-Петербург\\
                 2025}
  \end{center}

\end{frame}
}


% ====== СЛАЙД: ПЛАН ======
\begin{frame}{План выступления}
\begin{enumerate}
  \item Постановка задачи и данные.
  \item Краткое введение в Python-инструментарий.
  \item Теория: корреляция и линейная регрессия.
  \item Демонстрация:
  \begin{itemize}
    \item описательная статистика и корреляции;
    \item модель OLS, сводка, проверка гипотез;
    \item диагностика остатков, доверительные и предсказательные интервалы;
    \item краткая связь с ANOVA (F-тест).
  \end{itemize}
  \item Выводы и ограничения.
\end{enumerate}
\end{frame}


% ====== СЛАЙД: ПОСТАНОВКА ЗАДАЧИ ======
\begin{frame}{Постановка задачи}
\textbf{Задача:} изучить линейную связь между признаками и целевой переменной и построить модель для предсказания.\\[6pt]
\textbf{Пример:} прогноз цены по площади и числу комнат (синтетический набор данных, размер $n=200$).\\[6pt]
\textbf{Почему линейная регрессия и корреляция?}
\begin{itemize}
  \item простые модели;
  \item тесная связь с проверкой гипотез и доверительными интервалами;
  \item соответствуют темам курса (оценивание, гипотезы, корреляция, регрессия, ANOVA).
\end{itemize}
\end{frame}



% ====== СЛАЙД: ИНСТРУМЕНТ ======
\begin{frame}{Инструмент: Python для прикладной статистики}
\textbf{Язык:} Python 3.x \quad \textbf{Библиотеки:} \texttt{pandas}, \texttt{numpy}, \texttt{scipy}, \texttt{matplotlib}.\\[4pt]
\textbf{Почему Python:}
\begin{itemize}
  \item Широкая экосистема статистики и визуализации.
  \item Лёгкая репродукция: Jupyter/Colab/VS Code.
  \item Быстрый путь: данные $\rightarrow$ анализ $\rightarrow$ графики.
\end{itemize}
\textbf{Где запускать:} Jupyter Notebook/Colab/VS Code.
\end{frame}


\begin{frame}[fragile]{Установка и запуск окружения}
\textbf{Установка пакетов:}
\begin{lstlisting}[language=bash]
pip install pandas numpy scipy matplotlib scikit-learn
\end{lstlisting}
\textbf{Запуск Jupyter:}
\begin{lstlisting}[language=bash]
jupyter notebook   # или jupyter lab
\end{lstlisting}
\textbf{Структура демо:}
\begin{itemize}
  \item Импорт и загрузка данных (\texttt{sklearn.datasets.load\_diabetes()}).
  \item Описательная статистика, матрица корреляций.
  \item Pearson/Spearman: $r$, p-value, доверительные интервалы (Fisher z).
  \item Визуализация: scatter, тепловая карта.
\end{itemize}
\end{frame}


\begin{frame}{Виды зависимости (смысл корреляции)}
\begin{itemize}
  \item \textbf{Функциональная (детерминированная)}: $Y=f(X)$.
  \item \textbf{Стохастическая}: изменение $X$ влечёт изменение $Y$ по вероятностному закону.
  \item \textbf{Независимость}: $P\{X\in A, Y\in B\}=P\{X\in A\}P\{Y\in B\}$.
\end{itemize}
Корреляционный анализ изучает меры связи (ковариация, корреляция) и позволяет проверять гипотезы о наличии/отсутствии связи.
\end{frame}

\begin{frame}{Ковариация и корреляция (Пирсон)}
Ковариация: \[
\mathrm{Cov}(X,Y)=\E[XY]-\E X\,\E Y=\E[(X-\E X)(Y-\E Y)].
\]
Коэффициент корреляции Пирсона:
\[
\rho_{XY}=\frac{\mathrm{Cov}(X,Y)}{\sigma_X\sigma_Y}\in[-1,1].
\]
Свойства: инвариантность к сдвигу/масштабу; $\rho=\pm 1$ $\Leftrightarrow$ линейная зависимость $Y=\beta_0+\beta_1 X$.\\
Важно: \textbf{корреляция} — это сила \emph{линейной} связи (не причинность).
\end{frame}


\begin{frame}{Проверка гипотезы $H_0:\rho=0$ и доверительный интервал}
Пусть $r$ — выборочная корреляция, $n$ — размер выборки.\\[2pt]
\textbf{t-статистика:} \[
t=\frac{r\sqrt{n-2}}{\sqrt{1-r^2}}\ \sim\ t_{n-2}\ \text{при } H_0.
\]
Критерий: отвергаем $H_0$ при $|t|>t_{1-\alpha/2,n-2}$ или по p-value $<\alpha$.\\[6pt]
\textbf{Fisher z-преобразование} для CI $\rho$: $z=\tfrac12\ln\frac{1+r}{1-r}$,\\
$z\pm z_{1-\alpha/2}/\sqrt{n-3}\ \Rightarrow\ \rho=\frac{e^{2z}-1}{e^{2z}+1}$.
\end{frame}


\begin{frame}{Pearson vs. Spearman (когда какой)}
\begin{tabular}{@{}p{0.32\linewidth}p{0.28\linewidth}p{0.28\linewidth}@{}}
\toprule
 & \textbf{Pearson $r$} & \textbf{Spearman $\rho_s$} \\
\midrule
Связь & Линейная & Монотонная (по рангам) \\
Выбросы & Чувствителен & Робастнее \\
Данные & Интервальные/отношений & Порядковые/монотонные \\
\bottomrule
\end{tabular}
\end{frame}

\begin{frame}{Данные для демонстрации}
\textbf{Набор:} \texttt{sklearn.datasets.load\_diabetes()} (медицина, 442 наблюдения).\\
Непрерывные признаки (возраст, ИМТ, АД и др.), целевая \texttt{target} — прогресс заболевания.\\
Удобно для матрицы корреляций и парных анализов.
\end{frame}

\begin{frame}[fragile]{Импорт и подготовка}
\begin{lstlisting}[language=Python]
import numpy as np, pandas as pd, matplotlib.pyplot as plt
from scipy import stats
from sklearn.datasets import load_diabetes

data = load_diabetes()
X = pd.DataFrame(data.data, columns=data.feature_names)
y = pd.Series(data.target, name="target")
df = pd.concat([X, y], axis=1)

df.head()       # первые строки
df.describe()   # описательная статистика
\end{lstlisting}
\end{frame}

\begin{frame}[fragile]{Матрица корреляций (Pearson)}
\begin{lstlisting}[language=Python]
corr_pearson = df.corr(numeric_only=True)  # Pearson по умолчанию
print(corr_pearson.round(3))

plt.figure()
plt.imshow(corr_pearson, vmin=-1, vmax=1)
plt.colorbar(); plt.title("Correlation matrix (Pearson)")
plt.xticks(range(len(corr_pearson.columns)), corr_pearson.columns, rotation=90)
plt.yticks(range(len(corr_pearson.index)), corr_pearson.index)
plt.tight_layout(); plt.show()
\end{lstlisting}
Интерпретация: пары с $|r|\gtrsim 0.3$–$0.4$ — кандидаты на значимую линейную связь.
\end{frame}


\begin{frame}[fragile]{Pearson: $r$ и p-value для пары признаков}
\begin{lstlisting}[language=Python]
x = df["bmi"]        # индекс массы тела
y = df["target"]     # прогресс заболевания
r_xy, p_xy = stats.pearsonr(x, y)
print(f"Pearson r(bmi, target) = {r_xy:.3f}, p-value = {p_xy:.3g}")
\end{lstlisting}
Гипотеза $H_0{:}\ \rho=0$. Если $p<0.05$, связь статистически значима. Знак $r$ укажет направление связи.
\end{frame}

\begin{frame}[fragile]{Spearman (ранговая корреляция)}
\begin{lstlisting}[language=Python]
corr_spearman = df.corr(numeric_only=True, method="spearman")
print(corr_spearman.round(3))

r_s, p_s = stats.spearmanr(df["bmi"], df["target"])
print(f"Spearman rho(bmi, target) = {r_s:.3f}, p-value = {p_s:.3g}")
\end{lstlisting}
Когда лучше Spearman: выбросы, нелинейная \emph{монотонная} связь, порядковые шкалы.
\end{frame}

\begin{frame}[fragile]{Доверительный интервал для $\rho$ (Fisher z)}
\begin{lstlisting}[language=Python]
def fisher_ci(r, n, alpha=0.05):
    z = 0.5*np.log((1+r)/(1-r))
    z_se = 1/np.sqrt(n-3)
    z_crit = stats.norm.ppf(1-alpha/2)
    z_lo, z_hi = z - z_crit*z_se, z + z_crit*z_se
    to_r = lambda z: (np.exp(2*z)-1)/(np.exp(2*z)+1)
    return to_r(z_lo), to_r(z_hi)

n = len(df)
lo, hi = fisher_ci(r_xy, n)
print(f"95% CI for rho (bmi,target): [{lo:.3f}, {hi:.3f}]")
\end{lstlisting}
Интерпретация: интервал не пересекает 0 $\Rightarrow$ линейная связь статистически значима.
\end{frame}

\begin{frame}[fragile]{Визуализация: scatter-плоты}
\begin{lstlisting}[language=Python]
plt.figure()
plt.scatter(df["bmi"], df["target"])
plt.xlabel("bmi"); plt.ylabel("target")
plt.title("bmi vs target (scatter)")
plt.show()
\end{lstlisting}
Цель: оценить форму связи (линейная/нелинейная), наличие выбросов.
\end{frame}

% ====== APPLICATIONS / LIMITS ======
\begin{frame}{Применение и ограничения}
\textbf{Где полезно:}
\begin{itemize}
  \item Быстрый отбор признаков (feature screening) перед моделированием.
  \item Диагностика мультиколлинеарности (высокие межпризнаковые корреляции).
\end{itemize}
\textbf{Предосторожности:}
\begin{itemize}
  \item Pearson отражает \textbf{линейную} связь; Spearman — \textbf{монотонную}.
  \item Выбросы и неоднородная дисперсия искажают $r$ (используйте визуализацию).
  \item Корреляция $\neq$ причинность.
  \item Множественные проверки $\Rightarrow$ корректировки (Bonferroni/FDR).
\end{itemize}
\end{frame}

\begin{frame}{Итоги}
\begin{itemize}
  \item Корреляция — простая и наглядная мера связи.
  \item Python (\texttt{pandas}, \texttt{scipy}) даёт мгновенный расчёт $r$, p-value и CI.
  \item Визуализация обязательна для корректной интерпретации.
\end{itemize}
Нотбук с кодом включите в архив сдачи; слайды — PDF/PPTX.
\end{frame}

\begin{frame}{Контакты и материалы}
Код демо: ноутбук \texttt{correlation\_demo.ipynb}.\\
Пакеты: \texttt{pandas}, \texttt{numpy}, \texttt{scipy}, \texttt{matplotlib}, \texttt{scikit-learn}.\\
Курс: «Методы статистической обработки информации».
\end{frame}

\end{document}
