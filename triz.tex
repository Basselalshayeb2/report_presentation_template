\documentclass[aspectratio=169]{beamer}
\usepackage{amsmath, amssymb, amsfonts, mathtools}
\usepackage{booktabs, graphicx}
\usepackage{hyperref}
\usepackage{siunitx}
\usepackage{physics}
\usepackage{caption}
\usepackage{listings}
\usepackage{tikz}
\usetikzlibrary{arrows.meta,positioning,shapes,fit,calc}
\usepackage{pgfplots}
\pgfplotsset{compat=1.18}
  \newcommand{\E}{\mathbb{E}}
\newcommand{\Var}{\mathrm{Var}}

%Для защит онлайн лучше использовать разрешение 16x9
%\documentclass[aspectratio=169]{beamer}

\input{preamble.tex}

% То, что в квадратных скобках, отображается внизу по центру каждого слайда.
\title[TRIZ для голосового помощника врача]{Решение противоречия в дипломном проекте\\(голосовой помощник врача «СТОММИС») с применением ТРИЗ}

% То, что в квадратных скобках, отображается в левом нижнем углу.
\institute[СПбГУ]{\\ Курс ТРИЗ»}

% То, что в квадратных скобках, отображается в левом нижнем углу.
\author[Альшаеб Басель]{Альшаеб Басель, группа 24.М71-мм}

\begin{document}
{
\setbeamertemplate{footline}{}
% Лого университета или организации, отображается в шапке титульного листа
\begin{frame}
  \includegraphics[width=1.4cm]{pictures/SPbGU_Logo.png}
\vspace{-35pt}
\hspace{-10pt}
\begin{center}
   \begin{tabular}{c}
        \scriptsize{Санкт-Петербургский государственный университет} \\
        \scriptsize{Кафедра системного программирования}
    \end{tabular}
\titlepage
\end{center}

\btVFill

{\scriptsize
  % У научного руководителя должна быть указана научная степень
   \textbf{Преподаватель:} к.ф.-м.н. Старший преподаватель С.С. Сысоев \\
 }
\begin{center}
  \vspace{5pt}
  \scriptsize{Санкт-Петербург\\
                 2025}
  \end{center}

\end{frame}
}

% ====== Description of the task ======
\begin{frame}{Описание задачи}
  \begin{itemize}
    \item В существующей медицинской информационной системе \textbf{СТОММИС} врачи взаимодействуют с интерфейсом с помощью традиционных средств ввода — клавиатуры, мыши или сенсорного экрана.
    Эти методы могут быть неэффективны во время медицинских процедур или в ситуациях, где требуется быстрое и «безрукое» управление.
    \item \textbf{Цель данной работы:} усовершенствовать систему, внедрив интерфейс голосового управления, который позволит врачам выполнять ключевые действия при помощи голосовых команд.
    Такое решение повышает удобство, снижает нагрузку на пользователя и обеспечивает более быструю и безопасную работу с системой в клинических условиях.
  \end{itemize}
  \bigskip
\end{frame}

% ====== Description of the technical contradiction ======
\begin{frame}{Описание технического противоречия}
  \item Основная сложность заключается в проектировании подсистемы распознавания речи:
  \begin{itemsize}
    \item Непрерывное прослушивание обеспечивает высокую точность, но требует значительных вычислительных ресурсов.
    \item Ограниченное прослушивание экономит ресурсы, но может привести к потере важных голосовых команд.
  \end{itemsize}
  \bigskip
  \begin{itemize}
    \item Если слушаем \emph{всё} аудио, то растут расходы (время, вычисления, сеть).
    \item Если слушаем \emph{не всё}, то рискуем пропустить команду врача.
    \item Хотим: \textbf{минимальные ресурсы} и \textbf{высокая точность распознавания}.
  \end{itemize}
  \bigskip
  \textbf{Техническое противоречие:} Улучшение надёжности распознавания \(\Rightarrow\) рост энергопотребления/сложности.
\end{frame}



% ====== Parameters of TRIZ using the table of parameters ======
\begin{frame}{Параметры ТРИЗ}
    \begin{columns}[T,onlytextwidth]
      \begin{column}{0.45\textwidth}
        \begin{block}{Улучшаемый параметр}
          \textbf{Надёжность распознавания} (\#27)
        \end{block}
        \begin{block}{Ухудшаемые параметры}
          \begin{itemize}
            \item Энергопотребление (\#21)
            \item Сложность устройства/системы (\#36)
          \end{itemize}
        \end{block}
      \end{column}
      \begin{column}{0.55\textwidth}
        \centering
        \begin{tikzpicture}[font=\small]
          \node[draw,rounded corners,fill=blue!7,inner sep=6pt] (p27) {27: Надёжность};
          \node[draw,rounded corners,fill=red!7,inner sep=6pt,below left=9mm and -1mm of p27] (p21) {21: Энергопотр.};
          \node[draw,rounded corners,fill=red!7,inner sep=6pt,below right=9mm and -1mm of p27] (p36) {36: Сложность};

          \draw[-{Latex[length=2mm]}] (p27.south west) -- (p21.north east);
          \draw[-{Latex[length=2mm]}] (p27.south east) -- (p36.north west);
        \end{tikzpicture}
      \end{column}
    \end{columns}
\end{frame}

% === Matrix suggestions ===
\begin{frame}{Матрица противоречий: подсказки принципов}
  \begin{itemize}
    \item 27 $\uparrow$ vs 21 $\uparrow$ \(\Rightarrow\) \textbf{Принципы: 10, 19, 28, 35}
    \item 27 $\uparrow$ vs 36 $\uparrow$ \(\Rightarrow\) \textbf{Принципы: 1, 10, 28, 35}
  \end{itemize}
  \bigskip
  \begin{block}{Интерпретация для проекта}
  \begin{enumerate}
    \item \textbf{1. Дробление (Segmentation)} — разнести пайплайн на фазы.
    \item \textbf{10. Упреждающее действие (Preliminary action)} — лёгкий триггер до тяжёлой модели.
    \item \textbf{35. Изменение параметров (Parameter change)} — динамические частоты/буферы/модели.
    \item 19. Периодическое действие; 28. Замена механической системы (на ИИ-эвристику).
  \end{enumerate}
  \end{block}
  \end{frame}


\end{document}
