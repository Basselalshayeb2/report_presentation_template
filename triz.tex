\documentclass[aspectratio=169]{beamer}
\usepackage{amsmath, amssymb, amsfonts, mathtools}
\usepackage{booktabs, graphicx}
\usepackage{hyperref}
\usepackage{siunitx}
\usepackage{physics}
\usepackage{caption}
\usepackage{listings}
\usepackage{tikz}
\usetikzlibrary{arrows.meta,positioning,shapes,fit,calc}
\usepackage{pgfplots}
\pgfplotsset{compat=1.18}
  \newcommand{\E}{\mathbb{E}}
\newcommand{\Var}{\mathrm{Var}}

%Для защит онлайн лучше использовать разрешение 16x9
%\documentclass[aspectratio=169]{beamer}

\input{preamble.tex}

% То, что в квадратных скобках, отображается внизу по центру каждого слайда.
\title[TRIZ для голосового помощника врача]{Решение противоречия в дипломном проекте\\(голосовой помощник врача «СТОММИС») с применением ТРИЗ}

% То, что в квадратных скобках, отображается в левом нижнем углу.
\institute[СПбГУ]{\\ Курс ТРИЗ»}

% То, что в квадратных скобках, отображается в левом нижнем углу.
\author[Альшаеб Басель]{Альшаеб Басель, группа 24.М71-мм}

\begin{document}
{
\setbeamertemplate{footline}{}
% Лого университета или организации, отображается в шапке титульного листа
\begin{frame}
  \includegraphics[width=1.4cm]{pictures/SPbGU_Logo.png}
\vspace{-35pt}
\hspace{-10pt}
\begin{center}
   \begin{tabular}{c}
        \scriptsize{Санкт-Петербургский государственный университет} \\
        \scriptsize{Кафедра системного программирования}
    \end{tabular}
\titlepage
\end{center}

\btVFill

{\scriptsize
  % У научного руководителя должна быть указана научная степень
   \textbf{Преподаватель:} к.ф.-м.н. Старший преподаватель С.С. Сысоев \\
 }
\begin{center}
  \vspace{5pt}
  \scriptsize{Санкт-Петербург\\
                 2025}
  \end{center}

\end{frame}
}

% ====== Description of the task ======
\begin{frame}{Описание задачи}
  \begin{itemize}
    \item В существующей медицинской информационной системе \textbf{СТОММИС} врачи взаимодействуют с интерфейсом с помощью традиционных средств ввода — клавиатуры, мыши или сенсорного экрана.
    Эти методы могут быть неэффективны во время медицинских процедур или в ситуациях, где требуется быстрое и «безрукое» управление.
    \item \textbf{Цель данной работы:} усовершенствовать систему, внедрив интерфейс голосового управления, который позволит врачам выполнять ключевые действия при помощи голосовых команд.
    Такое решение повышает удобство, снижает нагрузку на пользователя и обеспечивает более быструю и безопасную работу с системой в клинических условиях.
  \end{itemize}
  \bigskip
\end{frame}

% ====== Description of the technical contradiction ======
\begin{frame}{Описание технического противоречия}
  \item Основная сложность заключается в проектировании подсистемы распознавания речи:
  \begin{itemsize}
    \item Непрерывное прослушивание обеспечивает высокую точность, но требует значительных вычислительных ресурсов.
    \item Ограниченное прослушивание экономит ресурсы, но может привести к потере важных голосовых команд.
  \end{itemsize}
  \bigskip
  \begin{itemize}
    \item Если слушаем \emph{всё} аудио, то растут расходы (время, вычисления, сеть).
    \item Если слушаем \emph{не всё}, то рискуем пропустить команду врача.
    \item Хотим: \textbf{минимальные ресурсы} и \textbf{высокая точность распознавания}.
  \end{itemize}
  \bigskip
  \textbf{Техническое противоречие:} Улучшение надёжности распознавания \(\Rightarrow\) рост энергопотребления/сложности.
\end{frame}



% ====== Parameters of TRIZ using the table of parameters ======
\begin{frame}{Параметры ТРИЗ}
    \begin{columns}[T,onlytextwidth]
      \begin{column}{0.45\textwidth}
        \begin{block}{Улучшаемый параметр}
          \textbf{Надёжность распознавания} (\#27)
        \end{block}
        \begin{block}{Ухудшаемые параметры}
          \begin{itemize}
            \item \textbf{Энергопотребление} (\#21)
            \item \textbf{Сложность устройства/системы} (\#36)
          \end{itemize}
        \end{block}
      \end{column}
      \begin{column}{0.55\textwidth}
        \centering
        \begin{tikzpicture}[font=\small]
          \node[draw,rounded corners,fill=blue!7,inner sep=6pt] (p27) {27: Надёжность};
          \node[draw,rounded corners,fill=red!7,inner sep=6pt,below left=9mm and -1mm of p27] (p21) {21: Энергопотр.};
          \node[draw,rounded corners,fill=red!7,inner sep=6pt,below right=9mm and -1mm of p27] (p36) {36: Сложность};

          \draw[-{Latex[length=2mm]}] (p27.south west) -- (p21.north east);
          \draw[-{Latex[length=2mm]}] (p27.south east) -- (p36.north west);
        \end{tikzpicture}
      \end{column}
    \end{columns}
\end{frame}

% === Matrix suggestions ===
\begin{frame}{Матрица противоречий}
  \begin{block}{27 (Надёжность) vs 21 (Энергопотребление)}
    Рекомендуемые принципы: \textbf{21, 11, 26, 31}
    \end{block}

    \begin{itemize}
      \item \textbf{21. Пропускание (Skipping)} — выполнять только важные участки процесса, игнорировать фоновые сигналы.
      \item \textbf{11. Резерв (Cushion in advance)} — предусмотреть буфер, предотвращающий потерю начала команды.
      \item \textbf{26. Копирование (Copying)} — использовать упрощённую «копию» основной модели для предварительного анализа.
      \item \textbf{31. Пористые структуры (Porous materials)} — чередовать интервалы активного и пассивного распознавания.
    \end{itemize}

    \bigskip
    \begin{block}{27 (Надёжность) vs 36 (Сложность)}
    Рекомендуемые принципы: \textbf{13, 35, 1}
    \end{block}

    \begin{itemize}
      \item \textbf{13. Обратная связь (The other way round)} — включать и выключать сложные компоненты в зависимости от состояния.
      \item \textbf{35. Изменение параметров (Parameter change)} — динамически менять частоту дискретизации и сложность модели.
      \item \textbf{1. Дробление (Segmentation)} — разделить систему на фазы: Idle → Trigger → Recognition.
    \end{itemize}
    \end{frame}

%     \begin{frame}{Выбранные изобретательские принципы и идея решения}
%       \begin{enumerate}
%         \item \textbf{26 – Копирование:} лёгкая нейросеть для предраспознавания (предварительный фильтр).
%         \item \textbf{21 – Пропускание:} игнорирование неинформативного аудио и фоновых шумов.
%         \item \textbf{11 – Резерв:} короткий буфер, предотвращающий потерю начала команды.
%         \item \textbf{35 – Изменение параметров:} адаптивная смена настроек (частота, параметры модели).
%         \item \textbf{1 – Дробление:} поэтапный конвейер обработки сигнала.
%         \item \textbf{13 – Обратная связь:} автоматическое включение/отключение сложных модулей в зависимости от активности.
%       \end{enumerate}

%       \bigskip
%       \textbf{Инвентивная идея:}
%       Создать адаптивную систему распознавания речи, в которой лёгкая модель предварительно фильтрует аудио и активирует полную нейросеть только при необходимости.
%       \end{frame}

%       \begin{frame}{Инвентивное решение — архитектура (диаграмма)}
%         \centering
%         \begin{tikzpicture}[
%           node distance=10mm,
%           box/.style={draw,rounded corners,align=center,minimum width=30mm,minimum height=8mm,fill=gray!10},
%           heavy/.style={draw,rounded corners,align=center,minimum width=38mm,minimum height=9mm,fill=blue!10,thick},
%           light/.style={draw,rounded corners,align=center,minimum width=38mm,minimum height=9mm,fill=green!10},
%           arr/.style={-{Latex[length=2mm]},thick}
%         ]
%         \node[box] (mic) {Микрофон / поток};
%         \node[light,right=18mm of mic] (vad) {Лёгкий VAD\\/ KWS (триггер)};
%         \node[heavy,right=18mm of vad] (asr) {Полная ASR\\(тяжёлая модель)};
%         \node[box,right=18mm of asr] (cmd) {Интерпретация\\команды};
%         \node[box,below=8mm of vad] (idle) {Режим Idle};
%         \node[box,below=8mm of asr] (rec) {Режим Recognition};

%         \draw[arr] (mic) -- (vad);
%         \draw[arr] (vad) -- node[above]{триггер} (asr);
%         \draw[arr] (asr) -- (cmd);

%         \draw[arr] (idle) -- (vad);
%         \draw[arr] (asr) |- ++(0,-8mm) -| (rec);
%         \draw[arr] (cmd) |- ++(0,-10mm) -| (idle);

%         \node[draw,dashed,rounded corners,fit=(idle)(vad),inner sep=6pt,label=below:\scriptsize Низкие параметры (частота, буфер, квант.)] {};
%         \node[draw,dashed,rounded corners,fit=(asr)(rec),inner sep=6pt,label=below:\scriptsize Высокие параметры (модель, частота, контекст)] {};
%         \end{tikzpicture}
%         \end{frame}

%         \begin{frame}{Диаграмма состояний (упрощённо)}
%           \centering
%           \begin{tikzpicture}[
%             st/.style={draw,rounded corners,minimum width=24mm,minimum height=7mm,align=center,fill=gray!10},
%             arr/.style={-{Latex[length=2mm]},thick},
%             node distance=22mm
%           ]
%           \node[st] (idle) {Idle\\(низк. ресурс)};
%           \node[st,right=of idle] (trig) {Trigger\\(VAD/KWS)};
%           \node[st,right=of trig] (rec) {Recognition\\(высок. ресурс)};

%           \draw[arr] (idle) -- node[above]{активность речи} (trig);
%           \draw[arr] (trig) -- node[above]{ключевое слово} (rec);
%           \draw[arr] (trig) |- +(-0.0,-1.8) -| node[below]{тайм-аут/шум} (idle);
%           \draw[arr] (rec) |- +(+0.0,-1.8) -| node[below]{конец команды} (idle);
%           \end{tikzpicture}
%           \end{frame}


%           % === Trade-off plot ===
% \begin{frame}{Компромисс «ресурсы–точность»: эффект адаптивности}
%   \centering
%   \begin{tikzpicture}
%   \begin{axis}[
%     width=0.9\linewidth,height=0.55\linewidth,
%     xlabel={Ресурсы (относит.)}, ylabel={Точность (относит.)},
%     xmin=0,xmax=1, ymin=0.7,ymax=1.0,
%     grid=both, legend style={at={(0.02,0.98)},anchor=north west}
%   ]
%   \addplot+[mark=*] coordinates {(0.85,0.96) (0.9,0.965)}; \addlegendentry{Непрерывная ASR}
%   \addplot+[mark=square*] coordinates {(0.25,0.8) (0.3,0.82)}; \addlegendentry{Слишком экономно}
%   \addplot+[mark=triangle*] coordinates {(0.35,0.9) (0.5,0.93) (0.6,0.95)}; \addlegendentry{Адаптивно (VAD\,$\to$\,ASR)}
%   \end{axis}
%   \end{tikzpicture}

%   \medskip
%   \small Адаптивная схема достигает \(\approx 60\text{–}70\%\) экономии ресурсов при точности \(90\text{–}95\%\).
%   \end{frame}

%   % === Evaluation slide ===
% \begin{frame}{Оценка эффекта}
%   \begin{itemize}
%     \item \textbf{Ресурсы:} снижение загрузки CPU/GPU и сетевого трафика на \(\mathbf{60\text{–}70\%}\).
%     \item \textbf{Точность:} поддержание уровня \(\mathbf{90\text{–}95\%}\) за счёт триггерного включения ASR.
%     \item \textbf{Стоимость:} меньше вызовов «тяжёлой» модели \(\Rightarrow\) меньшие расходы.
%   \end{itemize}
%   \end{frame}

%   % === Wrap up ===
%   \begin{frame}{Выводы}
%   \begin{enumerate}
%     \item Противоречие: точность \(\uparrow\) vs ресурсы/сложность \(\uparrow\).
%     \item Выбранные принципы ТРИЗ: \textbf{10}, \textbf{1}, \textbf{35}.
%     \item Инвентивное решение: адаптивный конвейер VAD/KWS \(\to\) ASR.
%   \end{enumerate}
%   \end{frame}

\end{document}
