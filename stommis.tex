\documentclass{beamer}
%Для защит онлайн лучше использовать разрешение 16x9
%\documentclass[aspectratio=169]{beamer}

\input{preamble.tex}

% То, что в квадратных скобках, отображается внизу по центру каждого слайда.
\title[Голосовые команды в МИС]{Разработка голосового помощника врача, интегрированного с медицинской информационной системой "СТОММИС", с использованием больших языковых моделей}

% То, что в квадратных скобках, отображается в левом нижнем углу.
\institute[СПбГУ]{Санкт-Петербургский государственный университет\\Кафедра системного программирования}

% То, что в квадратных скобках, отображается в левом нижнем углу.
\author[Альшаеб Басель]{Альшаеб Басель, группа 24.М71-мм}

\begin{document}
{
\setbeamertemplate{footline}{}
% Лого университета или организации, отображается в шапке титульного листа
\begin{frame}
  \includegraphics[width=1.4cm]{pictures/SPbGU_Logo.png}
\vspace{-35pt}
\hspace{-10pt}
\begin{center}
   \begin{tabular}{c}
        \scriptsize{Санкт-Петербургский государственный университет} \\
        \scriptsize{Кафедра системного программирования}
    \end{tabular}
\titlepage
\end{center}

\btVFill

{\scriptsize
  % У научного руководителя должна быть указана научная степень
   \textbf{Научный руководитель:} ст. преподаватель каф. системного программирования, к.ф.-м.н. С. С. Сысоев. \\
   \textbf{Площадка апробации:} СПб ГБУЗ «СП № 8»

 }
\begin{center}
  \vspace{5pt}
  \scriptsize{Санкт-Петербург\\
                 2025}
  \end{center}

\end{frame}
}


% ─────────────────────────────────────────────
% Intro / Problem
% ─────────────────────────────────────────────
\begin{frame}
  \frametitle{Проблема и мотивация}
  \begin{itemize}
    \item В стоматологии значимая часть данных должна быть занесена в электронную карту: жалобы, диагнозы, процедуры, материалы, рекомендации.
    \item В МИС «СТОММИС» исходный прототип голосового режима активировался \textbf{кнопкой} и записывал короткий фрагмент речи.
    \item Во время лечения врачу \textbf{неудобно взаимодействовать с компьютером}: заняты руки, перчатки, внимание на пациенте.
    \item Цель проекта: обеспечить \textbf{управление картой голосом прямо в процессе лечения} без ручного запуска записи.
  \end{itemize}
\end{frame}

% ─────────────────────────────────────────────
% Goal / Tasks
% ─────────────────────────────────────────────
\begin{frame}
  \frametitle{Постановка задачи}
  \textbf{Цель:} разработать ресурсоэффективный инструмент для анализа аудиопотока из браузера и выделения команд для МИС «СТОММИС».

  \vspace{8pt}
  \textbf{Задачи:}
  \begin{enumerate}
    \item Детекция участков с речью и «тишиной» в аудиопотоке (VAD) в браузере.
    \item Определение начала/конца командного сегмента (окончание команды --- пауза $\approx$2\,с).
    \item Экспериментировать с активацией голосового режима:
      \begin{itemize}
        \item KWS/wake word на клиенте;
        \item активация на сервере после ASR (ограниченный список команд и/или LLM-проверка).
      \end{itemize}
    \item Интеграция с серверной обработкой: транскодирование аудио, ASR, интерпретация команд, формирование действий для МИС.
    \item Первичная апробация и сравнение с базовым режимом (непрерывная обработка потока).
  \end{enumerate}
\end{frame}


\begin{frame}
  \frametitle{Ключевое противоречие}
  \begin{itemize}
    \item Для надежного распознавания команд хочется анализировать весь аудиопоток.
    \item Но непрерывная отправка/распознавание аудио:
    \begin{itemize}
      \item увеличивает нагрузку на сеть и сервер;
      \item увеличивает стоимость использования облачного ASR;
      \item повышает риск обработки нерелевантной речи (комментарии, шум).
    \end{itemize}
    \item Нужен механизм, который \textbf{минимизирует ресурсы}, но \textbf{не пропускает команды}.
  \end{itemize}
\end{frame}

% ─────────────────────────────────────────────
% Related / Market
% ─────────────────────────────────────────────
\begin{frame}
  \frametitle{Обзор существующих решений}
  \begin{itemize}
    \item Диктовка в медицинских системах: полный протокол приёма переводится в текст и сохраняется.
    \item Голосовые ассистенты общего назначения: хорошо понимают речь, но не привязаны к структуре медкарты и справочникам.
    \item KWS (wake word) в клиенте: локальная активация, но чувствительна к микрофону/шуму и часто требует обучения под пользователя.
  \end{itemize}
  \vspace{6pt}
  \textbf{Недостатки в нашем контексте:} нет гарантии \textit{структурированного} заполнения карты и сложно обеспечить устойчивую активацию при низком потреблении ресурсов.
\end{frame}

\begin{frame}
  \frametitle{Отличие от типовых решений}
  \begin{itemize}
    \item Мы не просто сохраняем речь: результатом являются \textbf{структурированные действия} в МИС.
    \item Команды интерпретируются и сопоставляются с \textbf{деревом сущностей} (диагнозы, процедуры, материалы) и полями карточки пациента.
    \item Предусмотрена фильтрация:
    \begin{itemize}
      \item \textbf{вход}: VAD + активация, чтобы отсекать «тишину» и нерелевантный поток;
      \item \textbf{выход}: ограничение допустимых команд и параметров (контроль формата ответа).
    \end{itemize}
  \end{itemize}
\end{frame}

% ─────────────────────────────────────────────
% Architecture
% ─────────────────────────────────────────────
\begin{frame}
  \frametitle{Архитектура решения}
  \begin{itemize}
    \item \textbf{Клиент (браузер):}
      \begin{itemize}
        \item захват микрофона (WebAudio);
        \item VAD (\texttt{@ricky0123/vad-web}) выделяет сегменты речи;
        \item Сегмент речи отправляется на сервер, где проверяется слово активации; при успехе запускается захват и обработка команды.
      \end{itemize}
    \item \textbf{Сервер (поликлиника):}
      \begin{itemize}
        \item приём аудио (\texttt{multipart/form-data}, поле \texttt{file});
        \item транскодирование в Ogg/Opus (FFmpeg);
        \item ASR (Yandex Speech) $\rightarrow$ текст;
        \item Проверка слова активации;
        \item интерпретация (LLM) $\rightarrow$ структурированная команда $\rightarrow$ действие в МИС.
      \end{itemize}
  \end{itemize}
\end{frame}

% ─────────────────────────────────────────────
% Wake word approach
% ─────────────────────────────────────────────
\begin{frame}
  \frametitle{Активация голосового режима: что пробовали}
  \begin{itemize}
    \item \textbf{Вариант 1 (браузер, KWS):} локальное распознавание \enquote{start}/\enquote{старт}.
    \begin{itemize}
      \item Плюс: не отправляет аудио на сервер до активации.
      \item Минусы: много \textbf{FN} (пропусков) в шумной среде, зависит от микрофона, требует \textbf{обучения/калибровки на каждом враче}.
    \end{itemize}
    \item \textbf{Вариант 2 (сервер, после ASR):} активация по распознанному тексту + ограниченный список команд / LLM-проверка.
    \begin{itemize}
      \item Устойчивее, не требует обучения, но зависит от качества ASR.
    \end{itemize}
  \end{itemize}

  \vspace{6pt}
  \textbf{В итоговой схеме} основной режим --- \textit{серверная} проверка активации; клиентская KWS оставлена как опциональная.
\end{frame}

% ─────────────────────────────────────────────
% Implementation details
% ─────────────────────────────────────────────
\begin{frame}[fragile]
  \frametitle{Технические детали: сервер пайплайн}
  \begin{itemize}
    \item \textbf{Вход:} аудиофайл сегмента команды из браузера (\texttt{multipart/form-data}, поле \texttt{file}).
    \item Приём аудио в Flask:
\begin{verbatim}
if "file" not in request.files: ...
raw = request.files["file"].read()
\end{verbatim}
    \item \textbf{Нормализация формата:} транскодирование в Ogg/Opus (FFmpeg) для стабильного ASR.
\begin{verbatim}
ffmpeg -i pipe:0 -c:a libopus -b:a 48k
\end{verbatim}
    \item \textbf{Распознавание речи (ASR):} Yandex Speech $\rightarrow$ транскрипция команды.
    \item \textbf{Интерпретация:} LLM преобразует текст в \textit{структурированное действие}.
    \item \textbf{Применение:} внесение изменений в карту пациента посредством \textbf{REST API} МИС «СТОММИС».
    \item ASR $\rightarrow$ текст $\rightarrow$ фильтрация/интерпретация:
      \begin{itemize}
        \item ограничение допустимых команд (whitelist);
        \item LLM (например, \texttt{qwen3-235b-a22b-fp8}) формирует JSON-результат с типом команды и аргументами.
      \end{itemize}
  \end{itemize}
\end{frame}

\begin{frame}[fragile]
  \frametitle{Задача 4: устойчивость к ошибкам ASR/LLM}
  \begin{itemize}
    \item Ограничение допустимых действий: \textbf{белый список интентов} и допустимых параметров.
    \item Жёсткий формат ответа LLM: \textbf{только JSON} по контракту \enquote{intent + slots + confidence}.
    \item Валидация перед выполнением: синтаксис JSON $\rightarrow$ схема $\rightarrow$ проверка слотов по справочникам.
    \item Fail-safe: при низкой уверенности или ошибке валидации возвращается \textbf{уточняющий вопрос}, действие не применяется.
  \end{itemize}
  \vspace{4pt}
\end{frame}

% ─────────────────────────────────────────────
% Experiment
% ─────────────────────────────────────────────
\begin{frame}
  \frametitle{Экспериментальное исследование}
  \begin{itemize}
    \item Площадка: тестовая среда и СПб ГБУЗ «СП № 8» (предварительная апробация).
    \item Сравнение двух режимов:
    \begin{enumerate}
      \item \textbf{Baseline:} непрерывная обработка аудиопотока/чанков.
      \item \textbf{Предложенный:} VAD в браузере + активация + отправка сегмента команды.
    \end{enumerate}
    \item Метрики:
    \begin{itemize}
      \item объём отправленного аудио;
      \item задержка \enquote{конец команды $\rightarrow$ действие в МИС};
      \item качество активации (FP/FN), качество сегментации.
    \end{itemize}
  \end{itemize}
\end{frame}

\begin{frame}
  \frametitle{Результаты (оценка на типовом сценарии)}
  \begin{columns}[T]
    \begin{column}{0.55\textwidth}
      \begin{itemize}
        \item За 10 минут приёма объём аудио на сервер существенно снижается: вместо всего потока отправляются только команды.
        \item Задержка выполнения команды остаётся на уровне $\approx$2--3\,с, но возникает \textbf{только по событию команды}.
        \item Снижается риск обработки нерелевантной речи.
      \end{itemize}
    \end{column}
    \begin{column}{0.45\textwidth}
      \centering
      \includegraphics[width=\linewidth]{audio_volume.pdf}
      \vspace{2pt}
      {\scriptsize Секунды аудио, отправленные на сервер (оценка).}
    \end{column}
  \end{columns}
\end{frame}

% ─────────────────────────────────────────────
% Conclusions
% ─────────────────────────────────────────────
\begin{frame}
  \frametitle{Результаты работы}
  \begin{itemize}
    \item Реализован клиентский модуль VAD в браузере для выделения речевых сегментов и окончания команды по паузе.
    \item Проработаны два варианта активации голосового режима (клиентский KWS и серверный контроль после ASR); выбран устойчивый режим без обучения на каждом враче.
    \item Реализован серверный конвейер: приём аудио, транскодирование, ASR (Yandex Speech), интерпретация команд LLM и формирование действий для МИС «СТОММИС».
    \item Проведено первичное сравнение с baseline и выполнена апробация в реальной поликлинике (результаты требуют дальнейшей стабилизации).
  \end{itemize}
\end{frame}

\begin{frame}
  \frametitle{Ограничения и планы}
  \begin{itemize}
    \item Чувствительность к шуму и микрофону остаётся важным фактором (особенно для клиентского KWS).
    \item Зависимость от сети/задержек при обращении к облачному ASR и LLM.
    \item Следующие шаги:
      \begin{itemize}
        \item расширение и формализация набора команд;
        \item стабилизация в полевых условиях и сбор метрик на большем объёме данных;
        \item оптимизация времени ответа (кэширование, локальные модели на сервере).
      \end{itemize}
  \end{itemize}
\end{frame}

% ─────────────────────────────────────────────
% Backup slide
% ─────────────────────────────────────────────
\appendix
\begin{frame}
  \frametitle{Дополнительно: таблица сравнения режимов}
  \scriptsize
  \begin{tabular}{|p{0.40\linewidth}|p{0.26\linewidth}|p{0.26\linewidth}|}
  \hline
  \textbf{Показатель} & \textbf{Baseline} & \textbf{VAD + активация} \\
  \hline
  Длительность аудио за 10 минут & $\approx$600\,с & $\approx$18--30\,с \\
  \hline
  Длительность распознаваемого фрагмента & 5--15\,с & 3--5\,с \\
  \hline
  Задержка \enquote{конец $\rightarrow$ действие} & распознавание идёт постоянно & распознавание запускается только после активации \\
  \hline
  Риск нерелевантной речи & высокий & сниженный \\
  \hline
  Требования к взаимодействию & нужен ручной контроль & управление голосом без рук \\
  \hline
  \end{tabular}
\end{frame}

\end{document}
