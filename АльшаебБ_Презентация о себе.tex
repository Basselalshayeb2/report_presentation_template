\documentclass{beamer}
%Для защит онлайн лучше использовать разрешение 16x9
%\documentclass[aspectratio=169]{beamer}

\input{preamble.tex}

% То, что в квадратных скобках, отображается внизу по центру каждого слайда.
\title[Короткая информация]{Презентация о себе - Короткая информация}

% То, что в квадратных скобках, отображается в левом нижнем углу.
\institute[СПбГУ]{\\ Курс "Управление проектами"}

% То, что в квадратных скобках, отображается в левом нижнем углу.
\author[Альшаеб Басель]{Альшаеб Басель, группа 24.М71-мм}

\begin{document}
{
\setbeamertemplate{footline}{}
% Лого университета или организации, отображается в шапке титульного листа
\begin{frame}
  \includegraphics[width=1.4cm]{pictures/SPbGU_Logo.png}
\vspace{-35pt}
\hspace{-10pt}
\begin{center}
   \begin{tabular}{c}
        \scriptsize{Санкт-Петербургский государственный университет} \\
        \scriptsize{Кафедра системного программирования}
    \end{tabular}
\titlepage
\end{center}

\btVFill

{\scriptsize
  % У научного руководителя должна быть указана научная степень
   \textbf{Научный руководитель:} Т. Димитри \\
 }
\begin{center}
  \vspace{5pt}
  \scriptsize{Санкт-Петербург\\
                 2025}
  \end{center}

\end{frame}
}

\begin{frame}[fragile]
  \frametitle{Основной технический стек}
  \begin{itemize}
    \item Я владею PHP (Laravel), JS (Node.js, Vue.js).
    \item Также имею опыт работы с Java (Spring Boot), Docker, SQL, MOM.
  \end{itemize}
\end{frame}

\begin{frame}[fragile]
  \frametitle{Опыт работы}
  Я работаю Software Engineer 5+ лет. В зоне моей ответственности:
  \begin{itemize}
    \item Разработка Backend- и Full-Stack-приложений.
    \item Проектирование архитектуры и API.
  \end{itemize}
\end{frame}

\begin{frame}[fragile]
  \frametitle{Описание решенной задачи}
  Одна из интересных задач, которую я решал, связана с чтением отпечатков пальцев с биометрического сканера и их синхронизацией с веб-сервером и мобильным приложением.
  \begin{itemize}
    \item Подключение сканера и получение биометрических данных Python/Flask.
    \item Разработка API для передачи данных на веб-сервер PHP/Laravel.
    \item Реализация механизма синхронизации с мобильным приложением в реальном времени FSM.
  \end{itemize}
\end{frame}


\begin{frame}[fragile]
  \frametitle{Видение профессионального роста}
  \begin{itemize}
    \item Через год: Работаю в сильной команде, углубляю знания в архитектуре ПО.
    \item Через три года: Руководитель технического направления или ведущий разработчик в крупном проекте или собственном стартапе.
    \item Через десять лет: Сооснователь или CTO успешного стартапа, связанного с программной инженерией или автоматизацией разработки.
  \end{itemize}
\end{frame}


\begin{frame}
  \frametitle{Ожидания от курса "Управление проектами"}
  \begin{itemize}
    \item Из этого курса я смогу увидеть более широкую картину разработки программного обеспечения и то, как должно начинаться программное обеспечение.
    \item Я хочу получить глубокое понимание процессов управления проектами, особенно в сфере agile-разработки, оценки рисков и управления требованиями.
    \item Также мне интересно, как эффективно управлять техническими командами.
  \end{itemize}
\end{frame}


\end{document}
